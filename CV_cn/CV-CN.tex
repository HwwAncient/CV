%----------------------------------------------------------------------------------------
%	PACKAGES AND OTHER DOCUMENT CONFIGURATIONS
%----------------------------------------------------------------------------------------

\documentclass[letterpaper,AutoFakeBold]{twentysecondcv} % a4paper for A4

\usepackage{booktabs}
\usepackage{fontspec}
\setmainfont[Mapping=tex-text]{KaiTi}
\usepackage{ctex}
% 设置双倍行距
\linespread{1.6}

%----------------------------------------------------------------------------------------
%	 PERSONAL INFORMATION
%----------------------------------------------------------------------------------------

% If you don't need one or more of the below, just remove the content leaving the command, e.g. \cvnumberphone{}

\profilepic{1.jpg} % Profile picture

\cvname{何万伟} % Your name
\cvjobtitle{武汉大学计算机学院} % Job title/career

\cvdate{软件工程专业\,2016\,级本科生} % Date of birth
\cvnumberphone{+86-15297986771} % Phone number
\cvsite{https://github.com/HwwAncient} % Personal website
\cvaddress{} % Short address/location, use \newline if more than 1 line is required
\cvmail{wanweihe@whu.edu.cn} % Email address

%----------------------------------------------------------------------------------------

\begin{document}

% 楷体
% \begin{CJK*}{UTF8}{gkai}
% 宋体

%----------------------------------------------------------------------------------------
%	 ABOUT ME
%----------------------------------------------------------------------------------------

% \aboutme{Alice is a sensible prepubescent girl from a wealthy English family who finds herself in a strange world ruled by imagination and fantasy. Alice feels comfortable with her identity and has a strong sense that her environment is comprised of clear, logical, and consistent rules and features. Alice's familiarity with the world has led one critic to describe her as a "disembodied intellect". Alice displays great curiosity and attempts to fit her diverse experiences into a clear understanding of the world.} % To have no About Me section, just remove all the text and leave \aboutme{}
\aboutme{
	何万伟,男,1999\,年\,2\,月\,5\,日生.
	\\[2ex]2016\,年\,9\,月进入武汉大学,就读于计算机学院软件工程专业.
	\\[2ex]曾获\href{https://raw.githubusercontent.com/HwwAncient/CV/master/Materials/16-17NationalScholarship.jpg}{2016-2017学年度国家奖学金}、\href{https://raw.githubusercontent.com/HwwAncient/CV/master/Materials/17-18NationalScholarship.jpg}
	{2017-2018学年度国家奖学金},并多次获得武汉大学三好学生、优秀团干、社会活动积极分子等荣誉.
	\\[2ex]现于\href{cs.whu.edu.cn/news_show.aspx?id=871}{小米-武汉大学人工智能联合实验室}、\href{http://sc.whu.edu.cn/}
{武汉大学语言与信息研究中心}实习,
	指导教师为\href{http://li3.whu.edu.cn/article/56.html}{彭敏}教授,研究课题包含机器阅读理解、开放域问答等.
	% V2:去掉人工智能
	\\[2ex]研究兴趣包括自然语言处理、深度学习、机器学习等,希望能够在研究生期间,投身人工智能、自然语言处理等相关领域.
}

%----------------------------------------------------------------------------------------

\makeprofile % Print the sidebar

\section {\large 本科教育}


\begin{center}
	\begin{tabular}{lll}
		\toprule
		\textbf{GPA} & \quad3.802 & \qquad \qquad 满绩:4.0 \\
		\midrule
		\textbf{平均分} & \quad90.263 & \qquad \qquad 满分:100 \\
		\midrule
		\textbf{专业排名} & \quad Top 6\% & \qquad \qquad 专业人数:240 \\
		\midrule
		\textbf{CET-6} & \quad523 & \qquad \qquad 过线分:425 \\
		\bottomrule
	\end{tabular}
\end{center}

% \begin{twenty}
% 	\twentyitem{GPA}{3.83}{}{}
% 	\twentyitem{Grades}{90.373}{}{}
% \end{twenty}

%----------------------------------------------------------------------------------------


\section{\large 获奖经历}


% TODO:加粗
\begin{itemize}
	\setlength{\itemsep}{0pt}
	\setlength{\parsep}{0pt}
	\setlength{\parskip}{0pt}
	\item
	\href{https://raw.githubusercontent.com/HwwAncient/CV/master/Materials/17-18NationalScholarship.jpg} 
	{2017\,-\,2018\,学年度国家奖学金(注:国家级,前1\%)}.
	\item 2017\,-\,2018\,学年度武汉大学三好学生(注:校级,前1\%).
	\item 2017\,-\,2018\,学年度武汉大学计算机学院优秀团干(注:院级,前1\%).
	\item 2018年“中国软件杯”双创大赛决赛三等奖.
	\item \href{https://raw.githubusercontent.com/HwwAncient/CV/master/Materials/16-17NationalScholarship.jpg}
	{2016\,-\,2017\,学年度国家奖学金(注:国家级,前1\%)}.
	\item 2016\,-\,2017\,学年度武汉大学三好学生(注:校级,前1\%).
	\item 2016\,-\,2017\,学年度武汉大学社会活动积极分子(注:校级,前1\%).
	\item 2015\,年全国高中生数学联赛二等奖.
\end{itemize}

%----------------------------------------------------------------------------------------


\section{\large 研究经历}


2018.03\,起 \qquad\qquad \href{cs.whu.edu.cn/news_show.aspx?id=871}{\emph{小米-武汉大学人工智能联合实验室}}\hfill 本科生实习
\\2018.01\,起 \qquad\qquad \href{http://sc.whu.edu.cn/}{\emph{武汉大学语言与信息研究中心}}\hfill 本科生实习
\begin{itemize}
	% \setlength{\itemsep}{0pt}
	% \setlength{\parsep}{0pt}
	% \setlength{\parskip}{0pt}
	\item 指导教师:\href{http://li3.whu.edu.cn/article/56.html}{彭敏}教授.
	\item 研究课题:机器阅读理解\,\emph{(Machine Reading Comprehension, MRC)}、
	智能问答系统\,\emph{(Intelligent Question Answering System, IQAS)}等.
	\item 研究方法:深度学习、机器学习、数据挖掘等.
\end{itemize}
2017.05-2018.12 \qquad \href{http://csold.whu.edu.cn/plus/view.php?aid=18}{\emph{智能化软件与服务研究所}}\hfill 本科生实习
\begin{itemize}
	% \setlength{\itemsep}{0pt}
	% \setlength{\parsep}{0pt}
	% \setlength{\parskip}{0pt}
	\item 指导教师:\href{http://cs.whu.edu.cn/teacherinfo.aspx?id=201}{贾向阳}导师.
	\item 研究课题:微服务\,\emph{(Micro Service)}.
\end{itemize}



%----------------------------------------------------------------------------------------

\section{\large 课程经历}


\begin{enumerate}
	\setlength{\itemsep}{0pt}
	\setlength{\parsep}{0pt}
	\setlength{\parskip}{0pt}
	\item 专业课:计算机网络、模式识别、软件工程、数据结构、操作系统、数据库、编译原理、离散数学、线性代数、高等数学、概率论、人工智能等.
	\item \emph{Coursera\,}在线课程:\href{https://www.coursera.org/specializations/deep-learning}{\emph{Deep Learning Specialization}}
		\begin{itemize}
			\item 该专项培训共分为:
			\emph{Neural Networks and Deep Learning},
			\emph{Improving Deep Neural Networks},
			\emph{Structuring Machine Learning Projects},
			\emph{Convolutional Neural Networks},
			\emph{Sequence Models}\,等五个课程.
			\item 该专项培训由\,\emph{deeplearning.ai}\,发布,主讲人为\,\href{https://www.coursera.org/instructor/andrewng}{\emph{Andrew Ng}}\,.
			\item 成功通过课程考试,并拿到\,\emph{Coursera}\,证书.
		\end{itemize}
\end{enumerate}

%----------------------------------------------------------------------------------------
\newpage % Start a new page
\makeprofile % Print the sidebar
%----------------------------------------------------------------------------------------




\section{\large 社会活动}


\begin{itemize}
	\setlength{\itemsep}{0pt}
	\setlength{\parsep}{0pt}
	\setlength{\parskip}{0pt}
	\item 2018.09\,-\,\qquad\qquad\quad 计算机学院2018级三班班主任助理.
	\item 2017.05\,-\,2018.05 \qquad 院学生会文艺部部长.
	\item 2017.05\,-\,2018.05 \qquad 院学生会舞蹈队队长.
\end{itemize}

%----------------------------------------------------------------------------------------

\section{\large 项目与比赛}

2019.02- \qquad \qquad \quad \emph{ 基于阅读理解的开放域QA系统设计 } \hfill 实习生
\begin{itemize}
	\setlength{\itemsep}{0pt}
	\setlength{\parsep}{0pt}
	\setlength{\parskip}{0pt}
	\item 
	该项目由小米-武汉大学人工智能联合实验室承办,用于小爱音箱,开放域问答.
	\item
	项目采用S-Net为主要模型,采用讯飞数据集,模型分为证据抽取和答案生成两大模块.
	\item 
	使用python和TesorFlow框架,个人目前完成了对讯飞数据集的数据增强工作和模型框架的搭建工作.
\end{itemize}

2018.09-2018.12 \qquad \emph{ 地面信息港服务集成开放平台 } \hfill 前后端开发者
\begin{itemize}
	\setlength{\itemsep}{0pt}
	\setlength{\parsep}{0pt}
	\setlength{\parskip}{0pt}
	\item 
	地面信息港是国家天地一体化信息网络的重要组成部分,由中国电力科学院承办.
	\item
	该平台旨在实现服务智能化组合与编排,以及时空数据的可视化展示.
	\item 
	整个项目采用微服务架构,基于Spring Cloud,同时联合了Activiti流程编排技术;使用Java作为后端开发语言、AngularJs作为前端开发语言.
	\item 
	个人完成了服务流程中人工任务的进行与提交,服务流程可视化展示等功能.
\end{itemize}

2017.09-2018.06 \qquad \emph{ 中国软件杯双创大赛 } \hfill 决赛三等奖
\begin{itemize}
	\setlength{\itemsep}{0pt}
	\setlength{\parsep}{0pt}
	\setlength{\parskip}{0pt}
	\item 
	参赛作品:小黑市-校园闲置物品交易共享平台.
	\item
	该项目旨在为在校大学生提供一套功能齐全、简单易用、可靠周到的二手物品交易平台,让校内没有难做的生意.
	\item 
	整个项目采用前后端分离的架构,前端基于微信小程序展示,后端采用Spring Boot技术定制Resful API接口.
	\item 
	个人完成了项目Resful API接口规范的制定,小程序架构的制定以及小程序页面的编写.
\end{itemize}


%----------------------------------------------------------------------------------------

\vspace{15pt}
\profilesection{}

\section{\large 附录}

\begin{itemize}
	\setlength{\itemsep}{0pt}
	\setlength{\parsep}{0pt}
	\setlength{\parskip}{0pt}
	\item 中英文成绩单:\href{https://raw.githubusercontent.com/HwwAncient/CV/master/Materials/Report.jpg}
	{武汉大学学生成绩单}
	\item 绩点证明:\href{https://raw.githubusercontent.com/HwwAncient/CV/master/Materials/GPAProof.jpg}
	{武汉大学全日制普通本科生学分绩点换算方法}
	\item \href{https://github.com/HwwAncient/CV/blob/master/Materials/Hww_Awards.pdf}
	{获奖证明}
\end{itemize}

%----------------------------------------------------------------------------------------
%	 SECOND PAGE EXAMPLE
%----------------------------------------------------------------------------------------

% \newpage % Start a new page

% \makeprofile % Print the sidebar

% \section{Other information}

%\subsection{Review}

%Alice approaches Wonderland as an anthropologist, but maintains a strong sense of noblesse oblige that comes with her class status. She has confidence in her social position, education, and the Victorian virtue of good manners. Alice has a feeling of entitlement, particularly when comparing herself to Mabel, whom she declares has a ``poky little house," and no toys. Additionally, she flaunts her limited information base with anyone who will listen and becomes increasingly obsessed with the importance of good manners as she deals with the rude creatures of Wonderland. Alice maintains a superior attitude and behaves with solicitous indulgence toward those she believes are less privileged.

%\section{Other information}

%\subsection{Review}

%Alice approaches Wonderland as an anthropologist, but maintains a strong sense of noblesse oblige that comes with her class status. She has confidence in her social position, education, and the Victorian virtue of good manners. Alice has a feeling of entitlement, particularly when comparing herself to Mabel, whom she declares has a ``poky little house," and no toys. Additionally, she flaunts her limited information base with anyone who will listen and becomes increasingly obsessed with the importance of good manners as she deals with the rude creatures of Wonderland. Alice maintains a superior attitude and behaves with solicitous indulgence toward those she believes are less privileged.

%----------------------------------------------------------------------------------------

\end{document} 